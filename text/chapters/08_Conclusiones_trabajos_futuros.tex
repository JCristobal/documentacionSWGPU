\chapter{Conclusiones y trabajos futuros}


\bigskip
\section{Conclusiones}
\bigskip

Una vez realizado el trabajo, se verán los objetivos (citados en el capítulo 2) y su respectivo avance.

\bigskip
Respecto a las clases para lanzar un algoritmo genético se han logrado cumplir, reuniéndolas en el archivo ejecutable \textit{geneticAlgorithm}. Se ha desarrollado como un módulo independiente, para poder ser usado por cualquier servicio, y no ser dependiente del servicio web, controlando la entrada de parámetros y control de fallos.

\bigskip
También se ha creado la estructura del servicio web, de manera que sea accesible desde cualquier servidor web.
Dicho servicio facilita la petición del algoritmo genético, y se obtendrán los resultados mediante el uso de la GPU del servidor.

\bigskip
Se a publicado el servicio, en forma de \textit{beta}. Se encuentra accesible en  \href{www.genmagic.ugr.es:8000}{www.genmagic.ugr.es:8000}, accesible dentro de la red de la UGR \cite{vpnugr}.
	

\section{Trabajos futuros}

\bigskip
Además de mantener y gestionar la versión actual del servicio web, se podría trabajar en la mejora de la experiencia de usuario.

\bigskip
También se refinará el algoritmo, logrando un mejor aprovechamiento de los recursos y mejorando los resultados.

\bigskip
Como se comenta en el capítulo 6 (en la subsección \textit{Ejecutable a usar por el BackEnd}) se estudiará la posibilidad de que el usuario pudiera añadir funcionalidades o cualquier procesamiento usando la GPU mediante CUDA. 

\bigskip
De esta manera se podría ejecutar cualquier opción de optimización, o fuera del ámbito de los algoritmos genéticos cualquier funcionalidad que requiera el usuario.

\bigskip
Esto requeriría un conocimiento mayor por parte del usuario. Además supondría un mayor consumo de recursos, pero con un buen uso del usuario y una correcta prevención de fallos se conseguiría una funcionalidad mayor y llena de posibilidades. 









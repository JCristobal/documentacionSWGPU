\chapter{Descripción del sistema}

\bigskip
En este capítulo se describe la fase de análisis del proyecto. Algunos requisitos o partes del capítulo fueron modificados o corregidos durante el trabajo. A continuación se muestran los definitivos.

\bigskip
\section{Análisis inicial}

Este proyecto fue propuesto desde un principio por {\tutor}, ya que en su ámbito de investigación requiere algoritmos evolutivos con muchos requerimientos de cómputo, por lo que al  pararelizarlos se logran unos resultados más rápidos, y al estar disponible en un servicio web no es necesario un dispositivo específico.

Después de la propuesta del trabajo, y antes del desarrollo del mismo se hicieron diversas reuniones para establecer sus bases y alcance. 

En la selección de tecnologías a usar se optó por la arquitectura de cómputo CUDA, ya que es actualmente la más potente, como se ve en el capítulo anterior, y se dispone de una tarjeta gráfica NVIDIA de gran potencia.

En la parte del servicio web se optó por el framework python Django, ya que se estudió con profundidad a lo largo del curso y es en la actualidad es la mejor opción para realizar un servicio web.

Dichas tecnologías son actuales, con una gran comunidad detrás, de manera que están en continuo desarrollo y mejora, además de estar bien asentadas, dando robustez al trabajo.


\newpage
\section{Objetivos}

Este trabajo tiene como principal objetivo el desarrollo de unas clases que lancen algoritmos genéticos de forma paralela, y puedan ser usados por un servicio web.

Los principales objetivos, de manera reducida, que se quieren alcanzar con este trabajo son:


\begin{itemize}
	\item  \textbf{OBJ. 1} Gestión del servicio web
	\begin{itemize}
		\item \textbf{OBJ. 1.1} Interfaz sencilla y adaptativa
		\item \textbf{OBJ. 1.2} Recogida de parámetros
		\item \textbf{OBJ. 1.3} Petición al servidor
		\item \textbf{OBJ. 1.4} Recogida y muestra de la solución
	\end{itemize}
		
	\item  \textbf{OBJ. 2} Gestión del algoritmo genético
	\begin{itemize}
		\item \textbf{OBJ. 2.1} Recogida de parámetros
		\item \textbf{OBJ. 2.2} Ejecución del algoritmo
		\item \textbf{OBJ. 2.3} Salida formateada
	\end{itemize}
	
\end{itemize}


	 

\newpage
\section{Resumen de implicados}

\bigskip
\textbf{Usuario}\\

Descripción: Representa al usuario que quiere hacer uso del sistema.\\

Tipo: Usuario del sistema\\

Responsabilidad: Completar el formulario con los datos que requiera el algoritmo genético.

\bigskip
\textbf{Administrador del sistema}\\

Descripción: Representa a la persona encargada de mantener, actualizar y gestionar el sistema.\\

Tipo: Superusuario\\

Responsabilidad: Realizar todas las actividades de gestión y actualización del sistema.



\newpage
\section{Especificación de requisitos}

\subsection{Requisitos funcionales}

\bigskip
En esta sección del capítulo se definirá y describiran las características de alto nivel (requisitos funcionales) del sistema que son necesarios para las necesidades del usuario.

\bigskip
\textbf{RF 1 Gestión del servicio web}

\begin{itemize}
	\item RF 1.1 Despliegue de la web con sus elementos y formularios bien situados.
	
	\item RF 1.2 Recogida de parámetros mediante el formulario de la función que ha escogido el usuario.
	
	\item RF 1.3 Petición para ejecutar el algoritmo 
	
	\item RF 1.4 Recogida y maquetado de la respuesta del algoritmo en la web
\end{itemize}

\textbf{RF 2 Gestión del algoritmo genético}

\begin{itemize}
	\item RF 2.1 Recogida de parámetros
	
	\item RF 2.2.1 Ejecución del algoritmo genético evaluado con Ackley
	
	\item RF 2.2.1 Ejecución del algoritmo genético evaluado con Rastrigin
	
	\item RF 2.3 Salida formateada
\end{itemize}






\subsection{Requisitos no funcionales}

\bigskip
En esta sección se establecen los requisitos no funcionales.

\bigskip
\textbf{Rendimiento}\\

\textbf{RNF 1}: Será implementado con tecnologías que optimicen el rendimiento y la eficacia, tanto en el servicio web como en el algoritmo genético.\\

\textbf{Disponibilidad}\\

\textbf{RNF 2}: Se intentará que el servicio esté disponible las 24 horas del día, con una estructura segura y preparada ante los fallos.\\


\textbf{Accesibilidad}\\

\textbf{RNF 3}: Se logrará una gran accesibilidad al tratarse de un servicio web.\\


\textbf{Usabilidad}\\

\textbf{RNF 4}: La interfaz, con el orden y disposición de los elementos será lo más usable posible.\\

\textbf{Interfaz}\\

\textbf{RNF 5}: La interfaz será sencilla e intuitiva\\

\textbf{RNF 6}: Se adaptará a cualquier tamaño de pantalla, tendrá un diseño adaptativo.\\


\textbf{Estabilidad}\\

\textbf{RNF 7}: Se buscará la mayor estabilidad en el sistema, usando tecnologías robustas y estables.\\


\textbf{Mantenimiento}\\

\textbf{RNF 8}: Se ha desarrollado de manera ordenada y documentando todos las partes, para que el mantenimiento pueda ser realizado por el mismo desarrollador o por terceros de manera sencilla.\\


\textbf{Soporte}\\

\textbf{RNF 9}: Se le facilitará al usuario contacto con alguien responsable del sistema, para que en caso de necesitar soporto le sea fácil y rápido.\\



\subsection{Requisitos de información}

\bigskip
En esta sección se establecen los requisitos de información, que estarán fuertemente relacionados a los requisitos funcionales, ya que será la información mínima para llevarlos a cabo.

\bigskip
\textbf{RI 1 Gestión del servicio web}

\begin{itemize}
	\item RI 1.1 Recogida de parámetros para el algoritmo genético

	\begin{itemize}
		\item Tamaño de la población
		\item Número de cromosomas
		\item Valor mínimo
		\item Valor máximo
		\item Probabilidad de cruce
		\item Probabilidad de mutación
		\item Número de generaciones
			\begin{itemize}
				\item RI 1.1.1 Recogida de parámetros para la función Ackley
				\begin{itemize}
					\item Valor A
					\item Valor B
					\item Valor C
				\end{itemize}
				
				\item RI 1.1.2 Recogida de parámetros para la función Rastrigin
				\begin{itemize}
					\item Valor A
				\end{itemize}
			\end{itemize}
	\end{itemize}

	\item RI 1.2. Recogida y maquetado de la salida del algoritmo genético
	\begin{itemize}
		\item Rendimiento
		\item Tiempo de cómputo
		\item Tamaño del cómputo
		\item Hebras usadas
		\item Datos de entrada (RI 1.1)
		\item Información del dispositivo que ha realizado el algoritmo
		\item Mejores variables obtenidas
		\item Mejor fitness
	\end{itemize}
	
\end{itemize}



\textbf{RI 2 Gestión del algoritmo genético}

\begin{itemize}
	\item RI 2.1 Recogida de parámetros 
	
	\begin{itemize}
		\item Dispositivo a usar
		\item Función de optimización a usar
		\item Tamaño de la población
		\item Número de cromosomas
		\item Valor mínimo
		\item Valor máximo
		\item Probabilidad de cruce
		\item Probabilidad de mutación
		\item Número de generaciones
		\begin{itemize}
			\item RI 2.1.1 Recogida de parámetros para la función Ackley
			\begin{itemize}
				\item Valor A
				\item Valor B
				\item Valor C
			\end{itemize}
			
			\item RI 2.1.2 Recogida de parámetros para la función Rastrigin
			\begin{itemize}
				\item Valor A
			\end{itemize}
		\end{itemize}
	\end{itemize}
	
	
	\item RI 2.2. Salida formateada en JSON de la solución obtenida
	\begin{itemize}
		\item Rendimiento
		\item Tiempo de cómputo
		\item Tamaño del cómputo
		\item Hebras usadas
		\item Datos de entrada (RI 2.1)
		\item Información del dispositivo que ha realizado el algoritmo
		\item Mejores variables obtenidas
		\item Mejor fitness
	\end{itemize}
	
\end{itemize}


\subsection{Restricciones}

\bigskip
Las restricciones del sistema serán las siguientes:

\bigskip
RI 1.1 Recogida de parámetros para el algoritmo genético:

\begin{itemize}
	\item RSTR 1.1: El tamaño de la población será menor de 10.000.
	\item RSTR 1.2: Habrá un máximo de 1.000 cromosomas (1000 variables).
	\item RSTR 1.3: El valor máximo tendrá que ser más grande el el valor mínimo.
	\item RSTR 1.4: La probabilidad de cruce tendrá que ser entre 0 y 1.
	\item RSTR 1.5: La probabilidad de mutación tendrá que ser entre 0 y 1.
\end{itemize}


\bigskip
RI 2.1 Recogida de parámetros (del algoritmo genético)

\begin{itemize}
	\item RSTR 2.1: Para escoger la función con la que optmizar introducimos 0 o 1.
	\item RSTR 2.2: El tamaño de la población será menor de 10.000.
	\item RSTR 2.3: Habrá un máximo de 1.000 cromosomas (1000 variables).
	\item RSTR 2.4: El valor máximo tendrá que ser más grande el el valor mínimo.
	\item RSTR 2.5: La probabilidad de cruce tendrá que ser entre 0 y 1.
	\item RSTR 2.6: La probabilidad de mutación tendrá que ser entre 0 y 1.
\end{itemize}


\newpage
\section{Diagramas del sistema}

\bigskip
\subsection{Diagramas de clases}
\bigskip

Aquí se verá en forma de diagramas de clases los elementos que conforman la estructura principal del sistema y como se relacionan entre ellos.

Los elementos del siguiente diagrama son:

\begin{itemize}
	\item Usuario
	\item Sistema
	\item Petición
	\item Salida del algoritmo
\end{itemize}


\bigskip
\begin{figure}[h]
	\centering
	\includegraphics[width=0.95\linewidth]{../images/diagrama_clases}
	\caption[Diagrama de clases del sistema]{Diagrama de clases del sistema}
	\label{fig:diagrama_clases}
\end{figure}

\bigskip
Los elementos citados tienen las relaciones:

\begin{itemize}
	\item Un \textit{Usuario} puede realizar 1 o varias \textit{Peticiones}.
	\item Una \textit{Petición} pertenece a un \textit{Usuario}.
	\item El \textit{Sistema} puede recibir 1 o varias \textit{Peticiones}.
	\item Una \textit{Petición} puede ser hecha al \textit{Sistema} (sólo uno).
	\item El \textit{Sistema} puede generar 1 o varias \textit{Salidas del algoritmo}.
	\item Cada \textit{Salida del algoritmo} se genera en (sólo un) \textit{Sistema}.
	\item Cada \textit{Petición} tiene una sola \textit{Salida del algoritmo}.
	\item Cada \textit{Salida del algoritmo} pertenece a una \textit{Petición}. 
	\item Cada \textit{Salida del algoritmo} tiene asociado a un \textit{Usuario}.
	\item Un \textit{Usuario} puede obtener 1 o varias \textit{Salidas del algoritmo}.
\end{itemize}


\bigskip
\subsection{Diagramas de caso de uso}
\bigskip

\bigskip
\textbf{Gestión del Servicio web}

\bigskip
En esta sección se verán las tareas que influyen en el servicio web. Este sistema hará posible que el usuario pueda realizar una petición de algoritmo genético con los parámetros que quiera.


\bigskip
\begin{figure}[h]
	\centering
	\includegraphics[width=0.9\linewidth]{../images/d_CU1}
	\caption[Diagrama de caso de uso en Gestión del Servicio web]{Diagrama de caso de uso en Gestión del Servicio web}
	\label{fig:d_CU1}
\end{figure}



\bigskip
\textbf{Gestión del algoritmo genético}

\bigskip
En esta sección se verán las tareas que influyen en el algoritmo genético. El sistema hará uso del algoritmo genético implementado para aprovechar la GPU, con los parámetros que se le pasen y generará una salida en formato JSON para poder trabajar con ella fácilmente.

\bigskip
\begin{figure}[h]
	\centering
	\includegraphics[width=0.9\linewidth]{../images/d_CU2}
	\caption[Diagrama de caso de uso en Gestión del algoritmo]{Diagrama de caso de uso en Gestión del algoritmo}
	\label{fig:d_CU2}
\end{figure}


\bigskip
\subsection{Flujo de la interfaz}
\bigskip

En esta sección se verá la navegación que se puede hacer a través de la web. Será una única pantalla, pero las distintas opciones se mostrarán en varias \textit{escenas} o \textit{pantallas}, para hacer un avance fluido y sin necesidad de retroceder o avanzar perdiendo información entre distintas pantallas.

\bigskip
\begin{figure}[h]
	\centering
	\includegraphics[width=1\linewidth]{../images/flujo_interfaz}
	\caption[Flujo de la interfaz web]{Flujo de la interfaz web}
	\label{fig:flujo_interfaz}
\end{figure}

